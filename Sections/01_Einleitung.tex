\section{Einleitung}

Echtzeitsysteme müssen innerhalb genauer Zeitvorgaben auf Ereignisse in der Umgebung reagieren.
Sie sind; reaktiv, effizient, verläslich, betriebssicher, spezifisch und real-time.

\subsection{Schichtmodell}

\begin{center}
	\includegraphics[width=.6\linewidth]{"Images/Schichtmodell.png"}
\end{center}

Das Betriebssystem im Schichtenmodell des Computersystems bietet der Anwendung API-Schnittstellen zur darunterliegenden realen Maschine und macht die Anwendung damit weitgehend unabhängig von der realen Maschine (CPU, Hardware, Peripherien,...).

\subsection{Entwicklungsprozess}

Unterschiedliche Möglichkeiten wie V-Modell und DevOps Lyfecycle (Loop).

\includegraphics[width=\linewidth]{"Images/UML-Diagram-types-1.png"}

\subsection{Mikrocontroller-Organisation}

\includegraphics[width=\linewidth]{"Images/MikrocontrollerOrganisation.png"}

\subsubsection{System Komponenten}

\begin{itemize}
	\itemsep-.5em 
	\item \textit{NVIC}: Nested Vectored Interrupt Controller, provides configurable interrupt handling abilities to the processor.
	\item \textit{SYSTICK}: System timer, 24-bit down-counter, automatic reload.
	\item \textit{MPU}: Memory Protection Unit, Zugriffsregeln für Privileged Access und User Program Access, wirft Exceptions.
	\item \textit{FPU}: Floating Point Unit, erweiterter Befehlssatz auf DSP.
\end{itemize}
